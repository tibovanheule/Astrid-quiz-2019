\documentclass{exam}
\usepackage[utf8]{inputenc}
\usepackage{graphicx}
\usepackage{dblfloatfix}
\makeatletter
\def\@seccntformat#1{%
  \expandafter\ifx\csname c@#1\endcsname\c@section\else
  \csname the#1\endcsname\quad
  \fi}
\makeatother
% naam punten
\pointname{ punten}
%hoogte header
\setlength\headheight{40pt} 
%header
\chead{
\begin{center}
\fbox{\fbox{\parbox{5.5in}{\centering Astrid quiz}}}
\end{center}
}
\cfoot{astrid, Master Key}



\begin{document}
\thispagestyle{empty}
\begin{titlepage}
\center
\includegraphics[scale=0.08]{astrid}
\linebreak
\linebreak
\linebreak

\vspace{1em}{\huge \bfseries  De Astrid Quiz Master Key }
\linebreak

\vspace{1em}{\LARGE  Door cultuur home Astrid} 
            \linebreak
\vspace{1em}{\large Geen toestemming tot herverdeling}
\end{titlepage}
\newpage
\section*{Algemeen}
Doorheen de volledige master key gelden volgende regels.
\large{ONTHOUD ZE, DANKU }
\begin{enumerate}
\item{Alle antwoorden zijn fonetisch (tenzij anders aangegeven), dit wil zeggen dat taalfouten niet meetellen.\par kash en cash worden beide juist gerekend}
\item{Alle nummers zijn exact, tenzij anders aangegeven.}
\item{Bij namen moeten zowel de voornaam en famillie naam fonetisch juist is, anders trek je 1 punt af per fout tot 0. }
\item{Een naamloos blad moet niet verbeterd te worden, dat is pech voor die groep. Verspil geen tijd met uitzoeken bij wie die hoort!}
\item{Een team die met het puntentabel vanonderen het antwoordblad speelt of vooraf invuld moet niet verbeterd te worden, dat is pech voor die groep.}
\end{enumerate}
\section*{Verbeterwijze}
De jury zal verdeeld worden in 2 groepen. De ene groep, de 2 personen aan de linkerkant van de tafel zal de bundel in twee splisten en verbeteren. Schrijf ook je eerste letter van je naam in de puntentabel, zo weten wie wat verbeterd heeft. Je geeft elk verbeterd blad door aan de controleurs, de tweede groep. één van deze twee zal de punten dan ook intikken op de computer.\linebreak\par

\large{VOOR VRAGEN TIJDENS DE QUIZ, STEL DEZE AAN TIBO EN NIET AAN DE PRESENTATOR LEMMY!}
\newpage
\section{Muziek Ronde}
\begin{enumerate}
\item{Op welke leeftijd is Micheal jackson gestorven?\par 50 jaar}
\item{Van welk land is de grootvader van Ludwig van Beethoven afkomstig? \par België, meer exact mechelen. Land moet vermeld worden, stad maakt niet uit.}
\item{Wat is de hoogst mogelijke stem van een vier stemmig koor?\par Sopraan}
\item{Welke muziek genre heeft een gelijkaardige naam als 1 van de planeten van ons zonnestelsel?\par Marsmuziek}
\item{Welke artiest kreeg de bijnaam "the man in black" wegens zijn opvallende kledingstijl van zijn tijd.\par Johnny cash}
\item{Op studio brussel hoor je jaarlijks de zwaarste lijst sinds 2009. de nummer 1 van deze lijst is de Amerikaanse band Metallica met het nummer 'Master of Puppets.' Ze stonden elk jaar al op nummer 1, behalve in 2014 en2015. Door welke band werden ze toen van hun troon gestoten? Gee ook de naam van het nummer en landvan herkomst.\par Channel zero, Black Fuel, BelgiË,  1 punt juist deel}
\item{ik ga de lyrics voorlezen van een nummer. Jullie geven de naam van dit nummer en de band.
Rolling stones, paint it black ? 1 punt per juist deel}
\item{ik ga de lyrics voorlezen van een nummer. Jullie geven de naam van dit nummer en de band.\par Eminem, The real slim shady}
\item{ naam lied, en band\par human, rag'n'bone man}
\item{ naam lied, en band\par Joan Jett and the blackhearts, I love Rock 'n' roll}

\end{enumerate}
\newpage
\section{videogames Ronde}
\begin{enumerate}

\item{Welke genre games beschrijf ik hier?: Deze soort games waren vooral populair in de jaren 80. Ze stonden vaak in vele rijen in game parken en men moest meestal rechtstaan op deze games te kunnen spelen.\\Arcade games}
\item{In welke Assassin’s Creed draait het allemaal rond een piraat te zijn? \\
Assassin’s Creed IV: Black Flag}
\item{In de games van Call of Duty zitten heel wat verwijzingen naar een film. Welke film is dit?\\ Saving private Bryan}
\item{Welke gameconsole is dit\\ Atari 2600 of ATARI VCS}
\item{Tegenwoordig is Minecraft weer een populair spel. De testversie van Minecraft dateert al van 2009. Maar in welk jaar kwam Minecraft officieel uit?\\ 2011}
\item{Farmville is een zeer bekend spelletje dat je op Facebook kan spelen. Maar wie is de ontwikkelaar van Farmville? \\ Zynga}
\item{In welke game komt deze easter egg voor.\\ just cause 3}
\item{Hoe noemt de stad onder de zee in de videogame Bioschock?\\ Rapture.}
\item{Grand Theft Auto V heeft de stad waarin je speelt gebaseerd op een Amerikaanse stad. Welke Amerikaanse stad is dit?\\ Los Angeles}
\item{Wie is deze pokémon?\\ Vulpix}

\end{enumerate}

\newpage
\section{Studio 100 ronde}
\begin{enumerate}
\item{Wat gaat Jonas doen wanner hij Spring verlaat\\ Hij wordt DJ in Ibiza}
\item{Wat zijn de voornamen van de originele K3?\\ Karen, Kristel, en Kathleen}
\item{Meneer de burgemeester uit Samson en Gert is in het echt ook een burgemeester. Van welke gemeente is dit?\\ Affligem}
\item{Hoe noemt deze vriend van Bumba?\\ Bumbalu}
\item{Studio 100 brengt naast films en tv-series ook musicals uit. Wat is de naam van hun musical die in februari 2020 live gaat?\\ Daens}
\item{Hoe noemt de acteur de mega Toby speelt?\\ Louis Talpe}
\item{Hoe noemt de attractie van het huis Anubis in Plopsaland de panne\\ “Anubis the ride”}
\item{In welke taal werd “Maya de bij” voor het eerst geschreven?\\ Duits}
\item{De acteur van meneer de burgemeester uit Samson en Gert speelt ook een rol in de serie Kabouter Plop. Welke personage is hij in deze serie?\\ Kabouter Plop}
\item{Met welk schip vaart Piet Piraat alle dagen uit?\\ De scheve schuit}
\end{enumerate}

\newpage
\section{Mythes en legendes}
\begin{enumerate}
\item{De oude Grieken zijn in de mythologie goed bekend met hun 12 Goden. Deze Goden hadden als verblijfplaats een berg. Wat is de naam van deze berg?\\ Olympus}
\item{Deze boom is onder de toeristen vrij goed bekend met zijn speciale vorm. Maar ook in de Afrikaanse mythologie speelt deze boom vaak een rol. Nu, wat is de naam van deze boom?\\ Baobab}
\item{In de Egyptische mythologie draaien de verhalen vaak rond een rivier. Welke rivier is dit?\\ De Nijl}
\item{Ook in Japan vind je vele mythes terug. Deze gaat over kami Inari, de God van de rijst en agricultuur. Deze God kan voorgesteld worden als een man of een vrouw en heeft een dier als boodschapper. Welk dier is dit?\\ Vos}
\item{Volgens de Noorse mythologie zijn er 9 werelden waar er leven op is. Deze werelden worden door de boom Yggdrasil verbonden met elkaar. Enkele voorbeelden hiervan zijn Asgaard, de wereld van de goden, en Jotunheim, de wereld van de reuzen. Maar wat is de naam van de wereld voor de mensen?\\ Midgaard}
\item{Welke Belgische Sage vertel ik hier?: De kwelgeest komt 's nachts tevoorschijn en achtervolgt de dronkaards. Eerst als een klein mannetje, maar hij kan zichzelf steeds groter en groter maken, tot hij boven de huizen uitsteekt. Eenmaal de dronkaards thuis in hun bed liggen klopt de kwelgeest op hun raam en bedreigt de dronkaards. Hij heeft ook een duivelse lag en komt eens voor in de stripboeken van Suske en Wiske.\\ Lange Wapper}
\newpage
\item{De Azteken werden door hun God naar een eiland geleid in het midden van een meer. Eenmaal ze hier toekwamen zagen ze iets eigenaardigs. Wat was dit? Het wordt ook afgebeeld op het wapenschild van een land.\\ Een adelaar die op een cactus zit en een slang opeet}
\item{Het hakenkruis is hier goed bekend doordat de Nazi's het gebruikten tijdens de Tweede Wereldoorlog. Maar er is ook een geloof waarin dit kruis wordt gebruikt. Welk geloof is dit?\\ Het hindoeïsme}
\item{Wie is de Romeinse tegenhanger van de oppergod Zeus? M.a.w. wie is de oppergod bij de Romeinen?\\ Jupiter}
\item{Het Nieuw-Zeelandse rugby team is bekend voor hun dansje dat ze altijd voor een wedstrijd opvoeren. Met dit dansje probeert men de goden aan te roepen. Wat is de naam van dit dansje?\\ Haka}
\end{enumerate}
\newpage
\section{Quizronde Gent}
\begin{enumerate}
\item{In Gent studeren meer dan 75000 studenten. Hiermee is Gent goed voor 30 percent van alle studenten aan een Vlaamse hogeschool of universiteit. Meer dan 44000 van de Gentse studenten studeren aan de UGent. Wat is met 8206 (cijfer: oktober 2018) studenten de grootste faculteit van onze universiteit? Voor het gemak staan alle faculteiten hieronder opgesomd zoals ze te vinden zijn op site van UGent.\\ Faculteit Geneeskunde en Gezondheidswetenschappen}
\item{Jeroen Meus beschrijft het in Dagelijkse Kost als “de Gentse trots”. Welk gerecht zoeken we?\\ (Gentse) Waterzooi}
\item{Voor de cantus freaks onder ons is dit een heel makkelijke vraag. Wat zijn de vier torens van Gent? De eerste toren is de universiteitsbibliotheek, een symbool van wetenschap, wijsheid en kennis. De tweede toren is de verblijfplaats van Het Lam Gods. Op de derde toren staat sinds 1377 een draak die de stad in de gaten houdt. De vierde toren is deel van een in Doornikse blauwe steen opgetrokken kerk en is een van de mooiste voorbeelden van de Scheldegotiek.\\ Boekentoren, Sint Baafs, Belfort en Sint Niklaas.       (alle vier juist=1 anders=0)}
\item{Deze zomer verruilde hij Slavia Praag voor AA Gent. Met 4,5 miljoen euro werd hij meteen de duurste inkomende transfer ooit. Wat is de naam van deze Kameroense verdediger?\\ (Michael) Ngadeu (Ngadjui)}
\item{Deze  Belgische film van Christophe Van Rompaeyuit uit 2008, gaat over een volkswijk in Gent. Hoofdpersonage Matty is moeder van drie kinderen. Haar man woont samen met een jongere vriendin, maar de beslissing om definitief te scheiden is nog niet genomen. Door een aanrijding op het parkeerterrein van de supermarkt komt Matty in contact met trucker Johnny, die wel wat in haar ziet. Wat is de naam van deze film?
Tip: de volkswijk doet denken aan een grote Russische stad.\\Aanrijding in Moscou}
\item{Antoon is een bekende Gentenaar. Hij groeide op in Aalst niet ver van de watertoren en zwierf daar al rond. Later verhuisde hij naar Gent waar hij al snel een bijnaam kreeg. Intussen doolt hij al meer dan 20 jaar door Gent. Hulp van buitenaf weigert de man steevast. Er werden al meerdere pogingen ondernomen om hem te laten begeleiden, zonder succes. Ondanks zijn slordig uiterlijk is hij een doodbrave man die niemand kwaad doet. In juli is er een toneelstuk van het Vernieuwd Gents Volkstoneel naar hem genoemd. Wat is de bijnaam van Antoon?\\ zakman}
\item{Een tijdje geleden is er in de Gentse kanaalzone een legionella uitbraak geweest. Legionella maakte 2 slachtoffers en 15 mensen werden zwaar ziek. De oorzaak van de uitbraak was een koeltoren van Stora Enso die slecht gereinigd werd. In welke gemeente ligt Stora Enso? Dit is ook de gemeente waarin legionella het meeste mensen besmette.\\ Evergem}
\item{Op de Gentse Groentemarkt staan twee kramen die hetzelfde product verkopen. Beide verkopers zijn meermaals agressief geweest tegen elkaar, hun vergunning is al eens twee weken ingetrokken, er zou al eens een kopstoot zijn uitgedeeld, een kraam zijn omgegooid, kwaad gesproken over elkaar, er gaan geruchten over diefstal tussen de twee verkopers en nog veel meer. Kortom een soort oorlog. Wat verkopen de mannen in hun kraam?\\ Neuzen (neuzekes) of cuberdons}
\item{In een kathedraal in Gent is het op hout geschilderde schilderij De aanbidding van het Lam Gods van de gebroeders Van Eyck te bezichtigen. Het panelengeheel kende doorheen de tijd een uiterst avontuurlijke geschiedenis en is tot op één paneel na intact te bewonderen. Welk paneel is niet te bewonderen?\\ De rechtvaardige rechters}

\item{Wie is de uitvoerder van volgend nummer?\\ Walter de Buck}
\end{enumerate}



\newpage
\input{natuur}
\newpage
\section{videogames Ronde}
\begin{enumerate}

\item{Welke genre games beschrijf ik hier?: Deze soort games waren vooral populair in de jaren 80. Ze stonden vaak in vele rijen in game parken en men moest meestal rechtstaan op deze games te kunnen spelen.\\Arcade games}
\item{In welke Assassin’s Creed draait het allemaal rond een piraat te zijn? \\
Assassin’s Creed IV: Black Flag}
\item{In de games van Call of Duty zitten heel wat verwijzingen naar een film. Welke film is dit?\\ Saving private Bryan}
\item{Welke gameconsole is dit\\ Atari 2600 of ATARI VCS}
\item{Tegenwoordig is Minecraft weer een populair spel. De testversie van Minecraft dateert al van 2009. Maar in welk jaar kwam Minecraft officieel uit?\\ 2011}
\item{Farmville is een zeer bekend spelletje dat je op Facebook kan spelen. Maar wie is de ontwikkelaar van Farmville? \\ Zynga}
\item{In welke game komt deze easter egg voor.\\ just cause 3}
\item{Hoe noemt de stad onder de zee in de videogame Bioschock?\\ Rapture.}
\item{Grand Theft Auto V heeft de stad waarin je speelt gebaseerd op een Amerikaanse stad. Welke Amerikaanse stad is dit?\\ Los Angeles}
\item{Wie is deze pokémon?\\ Vulpix}

\end{enumerate}

\newpage
\section{videogames Ronde}
\begin{enumerate}

\item{Welke genre games beschrijf ik hier?: Deze soort games waren vooral populair in de jaren 80. Ze stonden vaak in vele rijen in game parken en men moest meestal rechtstaan op deze games te kunnen spelen.\\Arcade games}
\item{In welke Assassin’s Creed draait het allemaal rond een piraat te zijn? \\
Assassin’s Creed IV: Black Flag}
\item{In de games van Call of Duty zitten heel wat verwijzingen naar een film. Welke film is dit?\\ Saving private Bryan}
\item{Welke gameconsole is dit\\ Atari 2600 of ATARI VCS}
\item{Tegenwoordig is Minecraft weer een populair spel. De testversie van Minecraft dateert al van 2009. Maar in welk jaar kwam Minecraft officieel uit?\\ 2011}
\item{Farmville is een zeer bekend spelletje dat je op Facebook kan spelen. Maar wie is de ontwikkelaar van Farmville? \\ Zynga}
\item{In welke game komt deze easter egg voor.\\ just cause 3}
\item{Hoe noemt de stad onder de zee in de videogame Bioschock?\\ Rapture.}
\item{Grand Theft Auto V heeft de stad waarin je speelt gebaseerd op een Amerikaanse stad. Welke Amerikaanse stad is dit?\\ Los Angeles}
\item{Wie is deze pokémon?\\ Vulpix}

\end{enumerate}

\newpage
\section{videogames Ronde}
\begin{enumerate}

\item{Welke genre games beschrijf ik hier?: Deze soort games waren vooral populair in de jaren 80. Ze stonden vaak in vele rijen in game parken en men moest meestal rechtstaan op deze games te kunnen spelen.\\Arcade games}
\item{In welke Assassin’s Creed draait het allemaal rond een piraat te zijn? \\
Assassin’s Creed IV: Black Flag}
\item{In de games van Call of Duty zitten heel wat verwijzingen naar een film. Welke film is dit?\\ Saving private Bryan}
\item{Welke gameconsole is dit\\ Atari 2600 of ATARI VCS}
\item{Tegenwoordig is Minecraft weer een populair spel. De testversie van Minecraft dateert al van 2009. Maar in welk jaar kwam Minecraft officieel uit?\\ 2011}
\item{Farmville is een zeer bekend spelletje dat je op Facebook kan spelen. Maar wie is de ontwikkelaar van Farmville? \\ Zynga}
\item{In welke game komt deze easter egg voor.\\ just cause 3}
\item{Hoe noemt de stad onder de zee in de videogame Bioschock?\\ Rapture.}
\item{Grand Theft Auto V heeft de stad waarin je speelt gebaseerd op een Amerikaanse stad. Welke Amerikaanse stad is dit?\\ Los Angeles}
\item{Wie is deze pokémon?\\ Vulpix}

\end{enumerate}

\newpage
\section{videogames Ronde}
\begin{enumerate}

\item{Welke genre games beschrijf ik hier?: Deze soort games waren vooral populair in de jaren 80. Ze stonden vaak in vele rijen in game parken en men moest meestal rechtstaan op deze games te kunnen spelen.\\Arcade games}
\item{In welke Assassin’s Creed draait het allemaal rond een piraat te zijn? \\
Assassin’s Creed IV: Black Flag}
\item{In de games van Call of Duty zitten heel wat verwijzingen naar een film. Welke film is dit?\\ Saving private Bryan}
\item{Welke gameconsole is dit\\ Atari 2600 of ATARI VCS}
\item{Tegenwoordig is Minecraft weer een populair spel. De testversie van Minecraft dateert al van 2009. Maar in welk jaar kwam Minecraft officieel uit?\\ 2011}
\item{Farmville is een zeer bekend spelletje dat je op Facebook kan spelen. Maar wie is de ontwikkelaar van Farmville? \\ Zynga}
\item{In welke game komt deze easter egg voor.\\ just cause 3}
\item{Hoe noemt de stad onder de zee in de videogame Bioschock?\\ Rapture.}
\item{Grand Theft Auto V heeft de stad waarin je speelt gebaseerd op een Amerikaanse stad. Welke Amerikaanse stad is dit?\\ Los Angeles}
\item{Wie is deze pokémon?\\ Vulpix}

\end{enumerate}

%eruit te halen
%\input{ronde1_min}
%\input{tafelronde1}
%\input{tafelronde2}
%\input{tussenronde3}

\end{document}